\documentclass{article}
\usepackage{lastpage}
\usepackage{ragged2e}

\usepackage{amsmath}%提供数学公式支持

\usepackage{graphics}%用于添加图片
\usepackage{graphicx}%加强插图命令
\newcommand{\figpath}[1]{contents/fig/#1}

\usepackage{fontspec}%用于配置字体
\usepackage[table]{xcolor}%用于各种颜色环境
\usepackage{enumitem}%用于定制list和enum
\usepackage{float}%用于控制Float环境,添加H参数(强制放在Here)
\usepackage[colorlinks,linkcolor=airforceblue,urlcolor=blue,anchorcolor=blue,citecolor=green]{hyperref}%用于超链接,另外添加该包目录会自动添加引用。

\usepackage[most]{tcolorbox}%用于添加各种边框支持
\usepackage[cache=true,outputdir=./out]{minted}%如果不保留临时文件就设置cache=false,如果输出设置了其他目录那么outputdir参数也有手动指定,否则会报错。
\tcbuselibrary{minted}%加载tcolorbox的代码风格

\usepackage[a4paper,left=3cm,right=3cm,top=3cm,bottom=3cm]{geometry}%用于控制版式
\usepackage{appendix}%用于控制附加文件
\usepackage{ifthen}

\usepackage{pdfpages}%用于支持插入其他pdf页
\usepackage{booktabs}%目前用于给表格添加 \toprule \midrule 等命令
\usepackage{marginnote} %用于边注
\usepackage[pagestyles,toctitles]{titlesec} %用于标题格式DIY
% \usepackage{fancyhdr}%用于排版页眉页脚
\usepackage{ragged2e} % 用于对齐
\usepackage{fixltx2e} %用于文本环境的下标
\usepackage{ulem} %用于好看的下划线、波浪线等修饰
\usepackage{pifont} %数学符号
\usepackage{amssymb} %数学符号

\usepackage{fontspec}
\setmainfont{DejaVu Serif}


\definecolor{langback}{RGB}{245,244,250}
\definecolor{langbacktitle}{RGB}{235,233,245}
\definecolor{langtitle}{RGB}{177,177,177}
\definecolor{langno}{RGB}{202,202,202}
\tcbset{arc=1mm}
\renewcommand{\theFancyVerbLine}{\sffamily\textcolor{langno}{\scriptsize\oldstylenums{\arabic{FancyVerbLine}}}}%重定义行号的格式
\newtcblisting{langbox}[1][tex]{%参考自https://reishin.me/tmux/ 的代码框样式
    arc=1mm,
    colframe=langbacktitle,
    colbacktitle=langbacktitle,
    coltitle=langtitle,
    fonttitle=\bfseries\sffamily,
    lefttitle=1mm,toptitle=0.5mm,bottomtitle=0.5mm,
    title = Code,
    drop shadow,
    listing engine=minted,
    minted style=colorful,
    minted language=#1,
    minted options={fontsize=\small,breaklines,autogobble,linenos,numbersep=2mm,xleftmargin=1mm},
    colback=langback,listing only,
    bottomrule=0mm,leftrule=0mm,toprule=0mm,rightrule=0mm,
    enhanced,
    % overlay={\begin{tcbclipinterior}\fill[langback] (frame.south west)rectangle ([xshift=5mm]frame.north west);\end{tcbclipinterior}}
}

\definecolor{boxback}{RGB}{245,246,250}
\newtcolorbox{markquote}{
    colback=boxback,fonttitle=\sffamily\bfseries,arc=0pt,
    boxrule=0pt,bottomrule=-1pt,toprule=-1pt,leftrule=-1pt,rightrule=-1pt,
    drop shadow,enhanced
}

\usepackage[UTF8,heading=true]{ctex}
\ctexset{
	section = {
	number = 第\chinese{section}节,
	format = \zihao{3}\bfseries,
	},
	subsection = {
	number = \arabic{section}.\arabic{subsection},
	format = \Large\bfseries
	},
	subsubsection = {
	number = \arabic{section}.\arabic{subsection}.\arabic{subsubsection},
	format = \Large\bfseries,
	},
    paragraph = {
	format = \large\bfseries,
	},
    subparagraph = {
	format = \large\bfseries,
	},
}

\setlength{\parindent}{2em}%设置首行缩进
\linespread{1.3}%设置行距

\setlength{\parskip}{0.5em}%设置段间距
\setcounter{tocdepth}{4}%设置目录级数
\setcounter{secnumdepth}{3}


\newtcbox{\inlang}[1][red]{on line,
arc=0pt,outer arc=0pt,colback=#1!10!white,colframe=#1!50!black,
boxsep=0pt,left=1pt,right=1pt,top=2pt,bottom=2pt,
boxrule=0pt,bottomrule=-1pt,toprule=-1pt,leftrule=-1pt,rightrule=-1pt}

\newlength\tablewidth


\definecolor{tablelinegray}{RGB}{221,221,221}
\definecolor{tablerowgray}{RGB}{247,247,247}
\definecolor{tabletopgray}{RGB}{245,246,250}
\definecolor{airforceblue}{rgb}{0.36, 0.54, 0.66}

\title{Cache 替换框架及算法实现}
\author{2017011313 刘丰源}

\begin{document}
\normalsize
\maketitle
\tableofcontents
\newpage









\section{实现内容概要}




\begin{enumerate}
\item
可修改参数的 cache 替换框架,包括缓存大小(cacheSize)、相联度(assoc)、块大小(lineSize)、替换算法(replPolicy)、写策略(writeAlloc)。以继承的形式实现替换算法,每个算法实现为独立的文件。
\item
实现了基础的 LRU、PLRU、RANDOM ,根据文献 Peress Y, Finlayson I, Tyson G, et al. CRC: Protected lru algorithm{[}C{]}. 2010. 实现了 PROTECT\_LRU 。
\item
根据输入的参数计算 tag 的长度,通过位操作,以 bit 为单位维护元数据,将所需内存降低至最小。
\item
LRU 的队列使用 uint\_8 数组保存,长度根据输入的参数确定,每个数据长度为  $\log_2(assoc)$  ,则数组长度为  $assoc * \log_2(assoc) / 8$  。
\item
PLRU 使用 bit binary tree 实现,也由 uint\_8 数组保存,数组长度为  $assoc / 8$  。
\end{enumerate}



\section{代码实现细节}




\subsection{运行代码}




\begin{itemize}
\item
\textbf{run.py}
\end{itemize}



用于运行框架的脚本,在文件中修改参数(相联度等)。然后执行 python3 run.py 即可运行。


\begin{itemize}
\item
\textbf{main.cpp}
\end{itemize}



主要用于读文件,处理文件内容,通过接口将参数传给 cache 替换框架。


\subsection{替换框架}




\begin{itemize}
\item
\textbf{utils.h}
\end{itemize}



一些工具函数,以及一些结构体的定义,cache 的元数据结构体。对于全相联的情况,tag 的长度为 64 ,对此情况进行了特判。部分代码如下:


\begin{langbox}[C++]
class LINE_STATE {
   private:
    u8* src;

   public:
    bool isValid() {
        if (tagLen < 63)
            return (*(ADDRINT*)(src)) >> 63;
        else
            return src[8] & 1;
    }

    bool isDirty() {
        if (tagLen < 63)
            return (*(ADDRINT*)(src)) >> 62;
        else
            return (src[8] >> 1) & 1;
    }

    void setValid(bool flag) {
        if (tagLen < 63) {
            if (flag)
                (*(ADDRINT*)(src)) |= (((ADDRINT)(1)) << 63);
            else
                (*(ADDRINT*)(src)) &= (~(((ADDRINT)(1)) << 63));
        } else {
            if (flag)
                src[8] |= 1;
            else
                src[8] &= (~(u8)(1));
        }
    }

    void setDirty(bool flag) {
        if (tagLen < 63) {
            if (flag)
                (*(ADDRINT*)(src)) |= (((ADDRINT)(1)) << 62);
            else
                (*(ADDRINT*)(src)) &= (~(((ADDRINT)(1)) << 62));
        } else {
            if (flag)
                src[8] |= 2;
            else
                src[8] &= (~(u8)(2));
        }
    }

    ...
};
\end{langbox}



\begin{itemize}
\item
\textbf{cache\_manager.{[}h|cpp{]}}
\end{itemize}



cache 替换框架的主要内容,主要的对外接口 LookupAndFillCache 接受两个参数:访存地址、访存类型。会处理参数为组 id ,传给替换算法,由算法决定哪块 cache 被替换。


主要用于保存 cache 的元数据,判断缓存是否命中。若命中,则调用算法的 UpdateReplacementState 接口,若未命中,则先调用算法的 GetVictimInSet 接口,然后再调用算法的 UpdateReplacementState 接口。针对每个函数的具体内容,以及每一行的作用,在代码中通过注释详细写出。


\subsection{基础替换算法}




\begin{itemize}
\item
\textbf{replacement\_state.h}
\end{itemize}



提供抽象类 CACHE\_REPLACEMENT\_STATE ,替换算法需要继承该类。子类需要实现 GetVictimInSet、UpdateReplacementState、InitReplacementState 三个纯虚函数。


\begin{itemize}
\item
\textbf{REPL\_RANDOM.h}
\end{itemize}



随机替换算法的实现非常简单:每次只需随机选择一块进行替换即可。这样的好处显而易见,由于没有维护任何额外的信息,也没有进行任何额外的计算,这样在额外空间消耗、信息维护用时、实现难度上有优势。另外。由于其随机性,它对各种故意构造的访问顺序具有一定的普适性。然而这种方法并没有考虑到访问时顺序的性质,也没有对访问的模式进行考虑。对于相同的程序, 其运行效率也会不稳定,这是其随机性造成的。


\begin{itemize}
\item
\textbf{REPL\_LRU.h}
\end{itemize}



首先需要一个按位的 LRU 队列,由于无法预先知道相联度,而元素的长度依赖于相联度,所以定义全局变量 lruElemNum 和 lruElemLen ,在算法初始化时给这两个全局变量赋值。


\begin{langbox}[C++]
u32 lruElemNum;
u32 lruElemLen;

REPL_LRU(u32 _sets, u32 _assoc, u32 _pol)
        : CACHE_REPLACEMENT_STATE(_sets, _assoc, _pol) {
        lruElemNum = _assoc;
        lruElemLen = FloorLog2(_assoc);
        if (lruElemLen == 0)
            lruElemLen = 1;
        InitReplacementState();
    }
\end{langbox}



创建 LRU 队列时根据这两个全局变量调整数组长度。LRU 的队列使用 uint\_8 数组保存,长度根据输入的参数确定,每个数据长度为  $\log_2(assoc)$  ,则数组长度为  $assoc * \log_2(assoc) / 8$  。部分代码如下:


\begin{langbox}[C++]
class LRUQueue {
   private:
    u8* queue;

   public:
    LRUQueue() {
        u32 len = (lruElemNum * lruElemLen + 7) / 8;
        if (len == 0)
            len = 1;
        queue = new u8[len];
    }

    u8 get_bit(u32 pos) { return (queue[pos / 8] >> (pos % 8)) & 1; }

    void set_bit(u32 pos, bool flag) {
        if (flag)
            queue[pos / 8] |= (((u8)(1)) << (pos % 8));
        else
            queue[pos / 8] &= (~(((u8)(1)) << (pos % 8)));
    }

    void set(u32 pos, u32 data) {
        for (u32 i = pos * lruElemLen; i < pos * lruElemLen + lruElemLen; ++i) {
            if (data & 1) {
                set_bit(i, true);
            } else
                set_bit(i, false);
            data >>= 1;
        }
    }

    ...
};
\end{langbox}



LRU 算法的实现中,每当需要替换的时候,都会找到最长时间未使用的块进行替换。具体而言,它维护了一个栈,每当一块被访问,就将其从栈底里去除一块并重新压栈。该算法充分利用了程序访问的局部性。如果程序访问的空间范围较小、缓存容量较大,这个算法的命中率将会较为可观。但是 LRU 无法处理数组大小大于 cache 容量的逐列扫描访问模式,缺失率将达到 100\% 。


\begin{itemize}
\item
\textbf{REPL\_PLRU.h}
\end{itemize}



PLRU 使用 bit binary tree 实现,也由 uint\_8 数组保存,数组长度为  $assoc / 8$  。相比于标准的 LRU 算法,其多了一定的随机性(某种程度上可以认为是简易的预测)。但是由于对于容量很大的 Cache,LRU 和 RANDOM 的命中率差别不大,所以 PLRU 的缺失率也和 LRU 接近。部分代码如下:


\begin{langbox}[C++]
class PLRUTree {
   private:
    u8* tree;

   public:
    PLRUTree() {
        i32 len = treeSize / 8;
        if (len == 0)
            len = 1;
        tree = new u8[len];
        memset(tree, 0, len * sizeof(u8));
    }

    void set_bit(i32 pos, bool flag) {
        if (flag) {
            tree[pos / 8] |= (((u8)(1)) << (pos % 8));
        } else {
            tree[pos / 8] &= (~(((u8)(1)) << (pos % 8)));
        }
    }

    void update(u32 wayID) {
        bool num[treeHeight];
        for (i32 i = 0; i < treeHeight; ++i) {
            num[i] = ((wayID & 1) == 1);
            wayID >>= 1;
        }
        i32 ind = 0;
        for (i32 i = treeHeight - 1; i >= 0; --i) {
            set_bit(ind, !num[i]);
            if (!num[i]) {
                ind = ind * 2 + 1;
            } else {
                ind = (ind + 1) * 2;
            }
        }
    }

    ...
};
\end{langbox}



\subsection{拓展替换算法}




\begin{itemize}
\item
\textbf{REPL\_PROTECT\_LRU.h}
\end{itemize}



参考文献:Peress Y, Finlayson I, Tyson G, et al. CRC: Protected lru algorithm{[}C{]}. 2010.


传统 LRU 只考虑了最近访问时间,没有考虑访问频率。PROTECT\_LRU 增加了对访问频率的考虑,其思路是:保护访问频率最高的 NUM\_MU 个路不被替换,剩下的路通过传统 LRU 算法进行替换。同时为了减小计算量,频率统计的大小不能超过 MAX\_COUNTER\_VAL 。如果达到该数值,则将所有路的访问次数除以二。


该算法可以有效缓解常用数据被替换出去的问题,但是也有新的问题:可能存在“曾经常用,但是后续不再被使用的数据”长期占据缓存块,导致有效缓存减少。因此需要适当调整的 NUM\_MU 和 MAX\_COUNTER\_VAL ,以达到更好的效果。


\section{缺失率分析}




\subsection{参数}




几种算法的结果都十分接近,以下以 lru 替换算法为例进行分析。


\begin{center}
\setlength\tablewidth{\dimexpr (\textwidth -8\tabcolsep)}
\arrayrulecolor{tablelinegray!75}
\rowcolors{2}{tablerowgray}{white}
\begin{tabular}{|p{0.345\tablewidth}<{\centering}|p{0.241\tablewidth}<{\centering}|p{0.207\tablewidth}<{\centering}|p{0.207\tablewidth}<{\centering}|}
\hline
\rowcolor{tabletopgray}
\textbf{ 组织方式\textbackslash{}块大小 }&\textbf{ 8     }&\textbf{ 32   }&\textbf{ 64   }\\
\hline
 直接映射        & 23.40 & 9.84 & 5.27 \\
\hline
 4 路组相联      & 23.28 & 9.63 & 5.01 \\
\hline
 8 路组相联      & 23.28 & 9.63 & 5.00 \\
\hline
 全相联映射      & 23.26 & 9.59 & 4.97 \\
\hline
\end{tabular}
\end{center}



\begin{itemize}
\item
\textbf{相联度}
\end{itemize}



从数据中可以看出,相联度对缺失率的影响并不大。增加相联度,相联度只会有轻微的降低(冲突减少)。替换算法和块大小才是影响缺失率的主要因素。


\begin{itemize}
\item
\textbf{块大小}
\end{itemize}



增加块大小,缺失率有明显降低。除了冲突减少,还得益于数据的局部性,每次以块为单位进行替换使得 cache 能够预取数据,因此缺失率有了明显降低。


\begin{itemize}
\item
\textbf{写分配}
\end{itemize}



以 8 路 8B 块分析写分配对缺失率的影响:写分配为 23.28 ,写不分配为 34.50 。由于局部性原理,刚写过的数据需要再次访问的可能性较大。写分配本质是就是在预测未来将要访问的数据。而由于实验给出的 cache 容量较大,使得写分配的预测结果准确率更高。


\begin{itemize}
\item
\textbf{写直达}
\end{itemize}



在真实机器中,出于一致性考虑,会定期将 cache 的 dirty 块数据写入 memory ,或者通过特定的指令进行同步。但是在实验中,并不考虑以上两种情况,脏页永远不会主动刷入内存,因此写直达和写回的结果一致。


\subsection{算法}




LRU 由于拥有较大的视野,对块使用的频繁程度有着精确 的估计,在大多数情况下都有着较低的缺失率。但实际上我们可以发现,每次更新缓存信息或替换块的时候,LRU 都需要遍历每个 assoc 来更新信息,即时间 复杂度为 O(assoc)。当 assoc 数量较大的时候该算法的时间消耗将会大大增加。


随机算法的时间复杂度是 O(1) 的,这是因为它每次都是随机选择一个块,并且在更新缓存信息的时候什么都不做,所以理论上应该是最快的。同时其缺失率与 LRU 接近,所以在是一个非常实用且常用的替换算法。


PLRU 缺失率与 LRU 接近,由于其算法具有一定的随机性和预测性,所以应该缺失率应该位于 RANDOM 和 LRU 之间。但是由于 RANDOM 和 LRU 的缺失率本身就十分接近,因此三者缺失率都很接近。


PROTECT\_LRU 在 cache 容量较小时,相比于另外三种算法有较好的表现,但是现代 CPU 都具有较大的 cache ,所以一般情况下优势并不明显。


\section{实验收获}




缓存替换算法是一个成本和效能、时间和空间、以及不同程序结构的多因素的权衡。通过这次实验,我了解到了各个缓存替换算法的大致思想,熟悉了几个基础算法的具体实现,对 cache 的理解也越发深刻了。(全相联的数据跑了一万年,让我深刻的意识到了该做法的不切实际)


感谢老师和助教给我这次提升和锻炼的机会。


\end{document}